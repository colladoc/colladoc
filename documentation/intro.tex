\section{Introduction}\label{sec:intro}
% section intro
Exploring the documentation APIs in large codebases can be challenging. Many tools have been developed to help managing the API documentation - Javadoc for Java; Sandcastle for .NET, RDoc for Ruby . Searching functionality is essential for these tools, especially in large codebases. 

Scaladoc is a tool for generating API documentation from Scala codebases. Collaborative Scaladoc (Colladoc) is an extension of Scaladoc which enables wiki-like editing of the API documentation. Although Colladoc is a promising upgrade of Scaladoc, it lacks useful search capability.

Implementing search functionality well has many high-level challenges - usability, good search performance, providing intuitive syntax and relevant results.

The aim of the Colladoc Smart Search project is to provide efficient, easy to use smart search for Colladoc. This extension provides full-text search as well as search for identifiers, symbols and documentation comments. A goal of the project is to support a powerful query syntax that resembles the Scala language.

The document is organized as follows: Section \ref{sec:spec} discuss the status of the planned project’s requirements. Section \ref{sec:design} gives brief overview of the project architecture. Section \ref{sec:implementation} explaining our main design and implementation decisions, Section \ref{sec:methodology} presents our planning strategy and development approaches, Section \ref{sec:groupwork} presents the division of work between the team members. Section \ref{sec:finalproduct} gives an overview of the project, outlines out main challenges and achievements, provide some concrete measures about the code quality and product performance. 